% \iffalse
\let\negmedspace\undefined
\let\negthickspace\undefined
\documentclass[journal,12pt,twocolumn]{IEEEtran}
\usepackage{cite}
\usepackage{amsmath,amssymb,amsfonts,amsthm}
\usepackage{algorithmic}
\usepackage{graphicx}
\usepackage{textcomp}
\usepackage{xcolor}
\usepackage{txfonts}
\usepackage{listings}
\usepackage{enumitem}
\usepackage{mathtools}
\usepackage{gensymb}
\usepackage{comment}
\usepackage[breaklinks=true]{hyperref}
\usepackage{tkz-euclide} 
\usepackage{listings}
\usepackage{gvv}  
\usepackage{tikz}
\usepackage{circuitikz} 
\usepackage{caption}

\def\inputGnumericTable{}                                
\usepackage[latin1]{inputenc}                 
\usepackage{color}                            
\usepackage{array}                            
\usepackage{longtable}                        
\usepackage{calc}                            
\usepackage{multirow}                      
\usepackage{hhline}                           
\usepackage{ifthen}                          
\usepackage{lscape}
\usepackage{amsmath}
\newtheorem{theorem}{Theorem}[section]
\newtheorem{problem}{Problem}
\newtheorem{proposition}{Proposition}[section]
\newtheorem{lemma}{Lemma}[section]
\newtheorem{corollary}[theorem]{Corollary}
\newtheorem{example}{Example}[section]
\newtheorem{definition}[problem]{Definition}
\newcommand{\BEQA}{\begin{eqnarray}}
\newcommand{\EEQA}{\end{eqnarray}}
\newcommand{\define}{\stackrel{\triangle}{=}}
\theoremstyle{remark}
\newtheorem{rem}{Remark}

\begin{document}
\title{Gate Assignment CH 31}
\author{Shravya Kantayapalam\\ EE23BTECH11030}
\maketitle

\begin{enumerate}
    \item \textbf{Question }:
The position \( x(t) \) of a particle, at constant \( \omega \), is described by the equation
\[
\frac{{d^2x}}{{dt^2}} = -\omega^2 x.
\]
The initial conditions are \( x(t=0) = 1 \) and \( \frac{{dx}}{{dt}}\bigg|_{t=0} = 0 \). 

The position of the particle at \( t = \frac{{3\pi}}{{\omega}} \) is \underline{\hspace{2cm}} (in integer).
\hfill{(GATE CH 2023)}
\solution

 \( x(t) \) 
 
  \( t = \frac{3\pi}{\omega} \).

\[ x(t) = A \sin(\omega t) + B \cos(\omega t) \]
where \( A \) and \( B \) are constants to be determined based on the initial conditions.

Given:
\[ x(0) = 1 \quad \text{and} \quad \frac{dx}{dt}\Bigg|_{t=0} = 0 \]

We have:
\[ x(0) = A \sin(0) + B \cos(0) = B = 1 \]
\[ \frac{dx}{dt} = A\omega\cos(\omega t) - B\omega\sin(\omega t) \]
\[ \frac{dx}{dt}\Bigg|_{t=0} = A\omega = 0 \]
From the second equation, \( A = 0 \).

Therefore, the solution to the differential equation is:
\[ x(t) = \cos(\omega t) \]

Now, we need to find the position of the particle at \( t = \frac{3\pi}{\omega} \). Substituting \( t = \frac{3\pi}{\omega} \) into \( x(t) \):
\[ x\left(\frac{3\pi}{\omega}\right) = \cos\left(\omega \cdot \frac{3\pi}{\omega}\right) = \cos(3\pi) = -1 \]

The position of the particle at \( t = \frac{3\pi}{\omega} \) is \(-1\) (in integer).


\end{enumerate}

\end{document}

